% -*- TeX-engine: xetex; eval: (auto-fill-mode 0); eval: (visual-line-mode 1); -*-
% Compile with XeLaTeX

%%%%%%%%%%%%%%%%%%%%%%%
% To do before class
%%%%%%%%%%%%%%%%%%%%%%%

% Send the Logistics/Week0Annoucnement (the night before).
% Send an email reminding students to bring a charged computer (the night before).

%%%%%%%%%%%%%%%%%%%%%%%
% Option 1: Slides: (comment for handouts)   %
%%%%%%%%%%%%%%%%%%%%%%%

%\documentclass[slidestop,compress,mathserif,12pt,t,professionalfonts,xcolor=table]{beamer}
%
%% solution stuff
%\newcommand{\solnMult}[1]{
%\only<1>{#1}
%\only<2->{\red{\textbf{#1}}}
%}
%\newcommand{\soln}[1]{\textit{#1}}
%
%\newcommand{\solnMultOn}[3]{
%\only<#1>{#3}
%\only<{#2}->{\red{\textbf{#3}}}
%}

%%%%%%%%%%%%%%%%%%%%%%%%%%%%%%%
% Option 2: Handouts, without solutions (post before class)    %
%%%%%%%%%%%%%%%%%%%%%%%%%%%%%%%

 \documentclass[11pt,containsverbatim,handout,xcolor=xelatex,dvipsnames,table]{beamer}

 % handout layout
 \usepackage{pgfpages}
 \pgfpagesuselayout{4 on 1}[letterpaper,landscape,border shrink=5mm]

 % solution stuff
 \newcommand{\solnMult}[1]{#1}
 \newcommand{\soln}[1]{}
 \newcommand{\solnMultOn}[3]{#3}

 % % This breaks things for me for some reason.
 % tell pgfpages how to set page sizes in XeLaTeX
\renewcommand\pgfsetupphysicalpagesizes{%
   \pdfpagewidth\pgfphysicalwidth\pdfpageheight\pgfphysicalheight%
}

%%%%%%%%%%%%%%%%%%%%%%%%%%%%%%%%%%%%
% Option 3: Handouts, with solutions (may post after class if need be)    %
%%%%%%%%%%%%%%%%%%%%%%%%%%%%%%%%%%%%

% \documentclass[11pt,containsverbatim,handout,xcolor=xelatex,dvipsnames,table]{beamer}

% % handout layout
% \usepackage{pgfpages}
% \pgfpagesuselayout{4 on 1}[letterpaper,landscape,border shrink=5mm]

% % solution stuff
% \newcommand{\solnMult}[1]{\red{\textbf{#1}}}
% \newcommand{\soln}[1]{\textit{#1}}

% % % This breaks things for me for some reason.
% % % tell pgfpages how to set page sizes in XeLaTeX
% % \renewcommand\pgfsetupphysicalpagesizes{%
% %    \pdfpagewidth\pgfphysicalwidth\pdfpageheight\pgfphysicalheight%
% % }

%%%%%%%%%%%%%%%%%%%%%%%%%%%%%%%
% Option 4: Notes Only
%%%%%%%%%%%%%%%%%%%%%%%%%%%%%%%

% % See http://tex.stackexchange.com/questions/114219/add-notes-to-latex-beamer
% \documentclass[10pt,containsverbatim,xcolor=xelatex,dvipsnames,table,notes=only]{beamer}

% % handout layout
% % \usepackage{pgfpages}
% % \pgfpagesuselayout{1 on 1}[letterpaper, landscape, border shrink=5mm]

% % solution stuff
% \newcommand{\solnMult}[1]{#1}
% \newcommand{\soln}[1]{}

% % % Having a problem with this.
% % tell pgfpages how to set page sizes in XeLaTeX
% % \renewcommand\pgfsetupphysicalpagesizes{%
% %   \pdfpagewidth\pgfphysicalwidth\pdfpageheight\pgfphysicalheight%
% %}

%%%%%%%%%%
% Load style file, defaults  %
%%%%%%%%%%

%%%%%%%%%%%%%%%%
% Themes
%%%%%%%%%%%%%%%%

% See http://deic.uab.es/~iblanes/beamer_gallery/ for mor options

% Style theme
\usetheme{Pittsburgh}

% Color theme
\usecolortheme{seahorse}

% Helvetica Neue Light for most text
\usepackage{fontspec}
\setsansfont{Helvetica Neue Light}

%%%%%%%%%%%%%%%%
% Packages
%%%%%%%%%%%%%%%%

\usepackage{geometry}
\usepackage{graphicx}
\usepackage{amssymb}
\usepackage{epstopdf}
\usepackage{amsmath}  	% this permits text in eqnarray among other benefits
\usepackage{url}		% produces hyperlinks
\usepackage[english]{babel}
\usepackage{colortbl}	% allows for color usage in tables
\usepackage{multirow}	% allows for rows that span multiple rows in tables
\usepackage{color}		% this package has a variety of color options
\usepackage{pgf}
\usepackage{calc}
\usepackage{ulem}
\usepackage{multicol}
\usepackage{textcomp}
\usepackage{listings}
\usepackage{changepage}
\usepackage{tikz}
\usetikzlibrary{trees}		% for probability trees
\usepackage{fancyvrb}	% for colored code chunks
\usepackage{nameref}

%%%%%%%%%%%%%%%%
% Remove navigation symbols
%%%%%%%%%%%%%%%%

\beamertemplatenavigationsymbolsempty
\hypersetup{pdfpagemode=UseNone} % don't show bookmarks on initial view

%%%%%%%%%%%%%%%%
% User defined colors
%%%%%%%%%%%%%%%%

% Pantone 2015 Spring colors
% http://iwork3.us/2014/09/16/pantone-2015-spring-fashion-report/
% update each semester or year

\xdefinecolor{custom_blue}{rgb}{0, 0.70, 0.79} % scuba blue
\xdefinecolor{custom_darkBlue}{rgb}{0.11, 0.31, 0.54} % classic blue
\xdefinecolor{custom_orange}{rgb}{0.97, 0.57, 0.34} % tangerine
\xdefinecolor{custom_green}{rgb}{0.49, 0.81, 0.71} % lucite green
\xdefinecolor{custom_red}{rgb}{0.58, 0.32, 0.32} % marsala

\xdefinecolor{custom_lightGray}{rgb}{0.78, 0.80, 0.80} % glacier gray
\xdefinecolor{custom_darkGray}{rgb}{0.54, 0.52, 0.53} % titanium

%%%%%%%%%%%%%%%%
% Template colors
%%%%%%%%%%%%%%%%

\setbeamercolor*{palette primary}{fg=white,bg= custom_blue}
\setbeamercolor*{palette secondary}{fg=black,bg= custom_blue!80!black}
\setbeamercolor*{palette tertiary}{fg=white,bg= custom_blue!80!black!80}
\setbeamercolor*{palette quaternary}{fg=white,bg= custom_blue}

\setbeamercolor{structure}{fg= custom_blue}
\setbeamercolor{frametitle}{bg= custom_blue!90}
\setbeamertemplate{blocks}[shadow=false]
\setbeamersize{text margin left=2em,text margin right=2em}

%%%%%%%%%%%%%%%%
% Styling fonts, bullets, etc.
%%%%%%%%%%%%%%%%

% title slide
\setbeamerfont{title}{size=\large,series=\bfseries}
\setbeamerfont{subtitle}{size=\large,series=\mdseries}
%\setbeamerfont{institute}{size=\large,series=\mdseries}

% color of alerted text
\setbeamercolor{alerted text}{fg=custom_orange}

% styling of itemize bullets
\setbeamercolor{item}{fg=custom_blue}
\setbeamertemplate{itemize item}{{{\small$\blacktriangleright$}}}
\setbeamercolor{subitem}{fg=custom_blue}
\setbeamertemplate{itemize subitem}{{\textendash}}
\setbeamerfont{itemize/enumerate subbody}{size=\footnotesize}
\setbeamerfont{itemize/enumerate subitem}{size=\footnotesize}

% styling of enumerate bullets
\setbeamertemplate{enumerate item}{\insertenumlabel.}
\setbeamerfont{enumerate item}{family={\fontspec{Helvetica Neue}}}
\setbeamerfont{enumerate subitem}{family={\fontspec{Helvetica Neue}}}
\setbeamerfont{enumerate subsubitem}{family={\fontspec{Helvetica Neue}}}

% make frame titles small to make room in the slide
\setbeamerfont{frametitle}{size=\small} 

% set Helvetica Neue font for frame and section titles
\setbeamerfont{frametitle}{family={\fontspec{Helvetica Neue}}}
\setbeamerfont{sectiontitle}{family={\fontspec{Helvetica Neue}}}
\setbeamerfont{section in toc}{family={\fontspec{Helvetica Neue}}}
\setbeamerfont{subsection in toc}{family={\fontspec{Helvetica Neue}}, size=\small}
\setbeamerfont{footline}{family={\fontspec{Helvetica Neue}}}
\setbeamerfont{subsection in toc}{family={\fontspec{Helvetica Neue}}}
\setbeamerfont{block title}{family={\fontspec{Helvetica Neue}}}

%%%%%%%%%%%%%%%%
% New fonts accessed by fontspec package
%%%%%%%%%%%%%%%%

% Monaco font for code
\newfontfamily{\monaco}{Monaco}

%%%%%%%%%%%%%%%%
% Color text commands
%%%%%%%%%%%%%%%%

%orange
\newcommand{\orange}[1]{\textit{\textcolor{custom_orange}{#1}}}

% green
\newcommand{\green}[1]{\textit{\textcolor{custom_green}{#1}}}

% red
\newcommand{\red}[1]{\textit{\textcolor{custom_red}{#1}}}

% dark gray
\newcommand{\darkgray}[1]{\textit{\textcolor{custom_darkGray}{#1}}}

% light gray
\newcommand{\lightgray}[1]{\textit{\textcolor{custom_lightGray}{#1}}}

% pink
\newcommand{\pink}[1]{\textit{\textcolor{pink}{#1}}}


%%%%%%%%%%%%%%%%
% Custom commands
%%%%%%%%%%%%%%%%

% empty box for probability tree frame
\newcommand{\emptybox}[2]{
	\fbox{ \begin{minipage}{#1} \hfill\vspace{#2} \end{minipage} }
}

% cancel
\newcommand{\cancel}[1]{%
    \tikz[baseline=(tocancel.base)]{
        \node[inner sep=0pt,outer sep=0pt] (tocancel) {#1};
        \draw[red, line width=0.5mm] (tocancel.south west) -- (tocancel.north east);
    }%
}

% degree
\newcommand{\degree}{\ensuremath{^\circ}}

% cite
\newcommand{\ct}[1]{
\vfill
{\tiny #1}}

% Note
\newcommand{\Note}[1]{
\rule{2.5cm}{0.25pt} \\ \textit{\footnotesize{\textcolor{custom_red}{Note:} \textcolor{custom_darkGray}{#1}}}}

% Remember
\newcommand{\Remember}[1]{\textit{\scriptsize{\textcolor{custom_red}{Remember:} #1}}}

% links: webURL, webLink
\newcommand{\webURL}[1]{\urlstyle{same}{\textit{\textcolor{custom_blue}{\url{#1}}}}}
\newcommand{\webLink}[2]{\href{#1}{\textcolor{custom_blue}{{#2}}}}

% mail
\newcommand{\mail}[1]{\href{mailto:#1}{\textit{\textcolor{custom_blue}{#1}}}}

% highlighting: hl, hlGr, mathhl
\newcommand{\hl}[1]{\textit{\textcolor{custom_blue}{#1}}}
\newcommand{\hlGr}[1]{\textit{\textcolor{custom_green}{#1}}}
\newcommand{\mathhl}[1]{\textcolor{custom_blue}{\ensuremath{#1}}}

% example
\newcommand{\ex}[1]{\textcolor{blue}{{{\small (#1)}}}}

% two col: two columns
\newenvironment{twocol}[4]{
\begin{columns}[c]
\column{#1\textwidth}
#3
\column{#2\textwidth}
#4
\end{columns}
}

% slot (for probability calculations)
\newenvironment{slot}[2]{
\begin{array}{c} 
\underline{#1} \\ 
#2
\end{array}
}

% pr: left and right parentheses
\newcommand{\pr}[1]{
\left( #1 \right)
}

%%%%%%%%%%%%%%%%
% Custom blocks
%%%%%%%%%%%%%%%%

% activity: less commonly used
\newcommand{\activity}[2]{
\setbeamertemplate{itemize item}{{{\small\textcolor{custom_orange}{$\blacktriangleright$}}}}
\setbeamercolor{block title}{fg=white, bg=custom_orange}
\setbeamerfont{block title}{size=\small}
\setbeamercolor{block body}{fg=black, bg=custom_orange!20!white!80}
\setbeamerfont{block body}{size=\small}
\begin{block}{Activity: #1}
\setlength\abovedisplayskip{0pt}
#2
\end{block}
}

% app: application exercise
\newcommand{\app}[2]{
\setbeamercolor{block title}{fg=white,bg=custom_green}
\setbeamercolor{block body}{fg=black,bg=custom_green!20!white!80}
\begin{block}{{\small Application exercise: #1}}
#2
\end{block}
}

% disc: discussion question
\newcommand{\disc}[1]{
\vspace*{-2ex}
\setbeamercolor{block body}{bg=custom_blue!25!white!80, fg=custom_blue!55!black!95}
\begin{block}{\vspace*{-3ex}}
#1
\end{block}
\vspace*{-1ex}
}

% clicker: clicker question
\newcommand{\clicker}[1]{
\setbeamercolor{block title}{bg=custom_blue!80!white!50,fg=custom_blue!30!black!90}
\setbeamercolor{block body}{bg=custom_blue!20!white!80,fg=custom_blue!30!black!90}
\begin{block}{\vspace*{-0.2ex}{\footnotesize Clicker question}\vspace*{-0.2ex}}
#1
\end{block}
}

% formula
\newcommand{\formula}[2]{
\setbeamercolor{block title}{bg=custom_blue!40!white!60,fg=custom_blue!55!black!95}
\begin{block}{{\small#1}}
#2
\end{block}
}

% code
\newcommand{\Rcode}[1]{
{\monaco {\footnotesize \textcolor{custom_darkBlue}{#1}}}
}

% output
\newcommand{\Rout}[1]{
{\monaco {\footnotesize \textcolor{custom_darkGray}{#1}}}
}

%%%%%%%%%%%%%%%%
% Change margin
%%%%%%%%%%%%%%%%

\newenvironment{changemargin}[2]{%
\begin{list}{}{%
\setlength{\topsep}{0pt}%
\setlength{\leftmargin}{#1}%
\setlength{\rightmargin}{#2}%
\setlength{\listparindent}{\parindent}%
\setlength{\itemindent}{\parindent}%
\setlength{\parsep}{\parskip}%
}%
\item}{\end{list}}

%%%%%%%%%%%%%%%%
% Footnote
%%%%%%%%%%%%%%%%

\long\def\symbolfootnote[#1]#2{\begingroup%
\def\thefootnote{\fnsymbol{footnote}}\footnote[#1]{#2}\endgroup}

%%%%%%%%%%%%%%%%
% Graphics
%%%%%%%%%%%%%%%%

\DeclareGraphicsRule{.tif}{png}{.png}{`convert #1 `dirname #1`/`basename #1 .tif`.png}

%%%%%%%%%%%%%%%%
% Slide number
%%%%%%%%%%%%%%%%

\setbeamertemplate{footline}{%
    \raisebox{5pt}{\makebox[\paperwidth]{\hfill\makebox[20pt]{\color{gray}
          \scriptsize\insertframenumber}}}\hspace*{5pt}}

          
%%%%%%%%%%%%%%%%
% Remove page numbers
%%%%%%%%%%%%%%%%

\newcommand{\removepagenumbers}{% 
  \setbeamertemplate{footline}{}
}

%%%%%%%%%%%%%%%%
% TOC slides
%%%%%%%%%%%%%%%%

\setbeamertemplate{section in toc}{\inserttocsectionnumber.~\inserttocsection}
\setbeamertemplate{subsection in toc}{$\qquad$\inserttocsubsectionnumber.~\inserttocsubsection \\}

\AtBeginSection[] 
{ 
  \addtocounter{framenumber}{-1} 
  % 
  {\removepagenumbers 
  {\small
    \begin{frame}<beamer> 
    \frametitle{Outline} 
    \tableofcontents[currentsection] 
  \end{frame} 
  } 
  }
} 

\AtBeginSubsection[] 
{ 
  \addtocounter{framenumber}{-1} 
  % 
  {\removepagenumbers 
  {\small
    \begin{frame}<beamer> 
    \frametitle{Outline} 
    \tableofcontents[currentsection,currentsubsection] 
  \end{frame} 
  } 
  }
}
\input{../../definitions_default.tex}
% ALT ALT
\input{../../definitions_custom.tex}

%%%%%%%%%%%
% Cover slide info    %
%%%%%%%%%%%

\title{Unit 4: Inference for numerical data}
\subtitle{3. ANOVA}
\author{Sta 101 - Spring 2015}
\date{March 2, 2015}
\institute{Duke University, Department of Statistical Science}

%%%%%%%%%%%
% Begin document   %
%%%%%%%%%%%

\newcommand{\mainideaA}{It is difficult to simultaneously compare many groups.}
\newcommand{\mainideaB}{ANOVA is useful for testing if there is \emph{some} difference
  between the means of many different groups.}
\newcommand{\mainideaC}{The test is based on comparing between group to within group variation.}

\begin{document}

%%%%%%%%%%%%%%%%%%%%%%%%%%%%%%%%%%%%

% Title Page

\begin{frame}[plain]

\titlepage
\vfill
{\scriptsize \webLink{\PersonalSite}{Dr. \LastName{}} \hfill Slides posted at  \webLink{\CourseSite}{\CourseSite}}
\addtocounter{framenumber}{-1} 

\end{frame}

%%%%%%%%%%%%%%%%%%%%%%%%%%%%%%%%%%%%

\section{Housekeeping}

%%%%%%%%%%%%%%%%%%%%%%%%%%%%%%%%%%%%

\begin{frame}
\frametitle{Announcements}

\begin{itemize}

\item 

\end{itemize}

\end{frame}

%%%%%%%%%%%%%%%%%%%%%%%%%%%%%%%%%%%%

\section{Main ideas}

%%%%%%%%%%%%%%%%%%%%%%%%%%%%%%%%%%%%

\subsection{\mainideaA}
\label{mi1}

%%%%%%%%%%%%%%%%%%%%%%%%%%%%%%%%%%%%

\begin{frame}
  \frametitle{1. mainideaA}

  \begin{center}
  {\Large NEWS FLASH!}

  \hspace{12pt}

  Jelly beans rumored to affect acne!!!
  \end{center}

  \pause

  \disc{
    How would you check this rumor? Imagine that doctors can assign an "acne
    score" to patients on a 0-100 scale.

    \begin{itemize}
      \item What would your research question be?
      \item How would you conduct your study?
      \item What statistical test would you use?
      \end{itemize}
    }

    \hfill \\

    \pause
   \soln{
     Use an independent samples t-test:
     \[
     H_0: \mu_{\textmd{jelly beans}} = \mu_{\textmd{placebo}}
     \]
     \[
     H_A: \mu_{\textmd{jelly beans}} \neq \mu_{\textmd{placebo}}
     \]
   }

\end{frame}

%%%%%%%%%%%%%%%%%%%%%%%%%%%%%%%%%%%%

\begin{frame}
  \frametitle{1. \mainideaA}

  \vfill
  
  \begin{center}
  \url{http://imgs.xkcd.com/comics/significant.png}
  \end{center}

\end{frame}

%%%%%%%%%%%%%%%%%%%%%%%%%%%%%%%%%%%%

\begin{frame}
  \frametitle{1. \mainideaA}

  \textbf{Suppose} $\alpha = 0.05$.

  \disc{What is the probability of correctly failing to reject
    \[
    H_0: \mu_{\textmd{purple}} = \mu_{\textmd{placebo}} \; ?
    \]
    }

    \pause

    \hfill \\

    \clicker{If no color of Jelly bean has any link to acne, what is the
      probability of making at least one type I error in the 20 trials?}
      \begin{enumerate}[(a)]
      \item 5\%
      \item 36\%
      \item \solnMultOn{2}{3}{64\%}
      \item 95\%
      \end{enumerate}


\end{frame}

%%%%%%%%%%%%%%%%%%%%%%%%%%%%%%%%%%%%

\subsection{\mainideaB}
\label{mi2}

%%%%%%%%%%%%%%%%%%%%%%%%%%%%%%%%%%%%

\begin{frame}
  \frametitle{2. \mainideaB}

Null hypothesis for "F-Test" (the test associated with ANOVA):
\[
H_0: \mu_{\textmd{placebo}} = \mu_{\textmd{purple}} = \mu_{\textmd{brown}} =
\ldots = \mu_{\textmd{peach}} = \mu_{\textmd{orange}} .
\]

\pause

\clicker{Which of the following is a correct version of the alternative
  hypothesis?}

\begin{enumerate}[(a)]
\item For any two groups, including the placebo group, no two group means are the same.
\item For any two groups, not including the placebo group, no two group means are the
  same.
\item Amongst the jelly bean groups, there are at least two groups that have different group means.
\item \solnMultOn{2}{3}{Amongst all groups, there are at least two groups that have different group means.}
\end{enumerate}

\end{frame}

%%%%%%%%%%%%%%%%%%%%%%%%%%%%%%%%%%%%

\begin{frame}
  \frametitle{2. \mainideaB}

The practical implication of this alternative is: ``At least one color of
jelly bean is linked to acne.''

\pause

\hfill \\

At least two of the group means are not the same: \pause
\begin{enumerate}
\item $\mu_{\textmd{placebo}} \neq \mu_{\textmd{color}}$ for some color of jelly
  bean, or \pause
\item $\mu_{\textmd{A}} \neq \mu_{\textmd{B}}$ for two colors, A and B. \pause

\vspace{6pt}

Then 

\vspace{3pt}

\begin{enumerate}
\item $\mu_{\textmd{A}} \neq \mu_{\textmd{placebo}}$, or \pause

\item $\mu_{\textmd{A}} = \mu_{\textmd{placebo}}$.  Thus, $\mu_{\textmd{B}} \neq
  \mu_{\textmd{A}} = \mu_{\textmd{placebo}}$.
\end{enumerate}

\end{enumerate}

\end{frame}

%%%%%%%%%%%%%%%%%%%%%%%%%%%%%%%%%%%%

\subsection{\mainideaC}
\label{mi3}

%%%%%%%%%%%%%%%%%%%%%%%%%%%%%%%%%%%%

\begin{frame}
  \frametitle{3. \mainideaC}

\centering
\(
\sum {\color{blue} \mid}^2 \Big/ \sum {\color{green} \mid}^2
\)

  \includegraphics[scale=0.6]{figures/anova-middle-ground-jelly-bean.png}

\end{frame}

%-------------------------------------------------------------------------------
\begin{frame}
\frametitle{Relatively large WITHIN group variation: little apparent difference}

\centering
\(
\sum {\color{blue} \mid}^2 \Big/ \sum {\color{green} \mid}^2
\)

    \includegraphics[scale=0.6]{Figures/anova-lots-of-within-group-jelly-bean.png}

\end{frame}

%-------------------------------------------------------------------------------
\begin{frame}
\frametitle{Relatively large BETWEEN group variation: there may be a difference}

\centering
\(
\sum {\color{blue} \mid}^2 \Big/ \sum {\color{green} \mid}^2
\)
    
    \includegraphics[scale=0.6]{Figures/anova-lots-of-between-group-jelly-bean.png}

\end{frame}

%%%%%%%%%%%%%%%%%%%%%%%%%%%%%%

\begin{frame}
  \frametitle{3. The F-test is based on comparing between group to within group variation}

For historical reasons, we use a modification of this ratio called the $F$-statistic:
\[
F = \frac{\sum {\color{blue} \mid}^2 \; \; / \; \; (j-1)}{\sum {\color{green} \mid}^2 \; \; / \; \; (n-j)}
\hspace{12pt} = \hspace{12pt} \frac{MSG}{MSE}.
\]

$j$: \# of groups; $n$: \# of obs.

\pause

\begin{center}
\small
\begin{tabular}{lllrrr}
  \hline
  & Df  & Sum Sq & Mean Sq & F value & Pr($>$F) \\ 
  \hline
  Between & j-1 & $\sum {\color{blue}\mid}^2$  & MSG & $F_{obs}$ & $p_{obs}$ \\ 
  Within  & n-j & $\sum {\color{green}\mid}^2$ & MSE &		 &  \\ 
   \hline
Total			& n-1 & $\sum ({\color{blue}\mid} + {\color{green}\mid})^2$    &                &                &
\end{tabular}
\end{center}

\pause

\[
p_{obs} = p(W > F_{obs} \mid H_0) \pause = p(W > F_{obs} \mid W \sim \textmd{F-dist}_{j-1, n-j})
\]

\centering
\url{bitly.com/dist_calc}

\end{frame}

%%%%%%%%%%%%%%%%%%%%%%%%%%%%%%

\begin{frame}

\centering
\includegraphics[scale=0.6,angle=270]{figures/cartoon.png}

\end{frame}

%%%%%%%%%%%%%%%%%%%%%%%%%%%%%%

\begin{frame}

Say $\alpha = 0.05$.

\clicker{What is the most accurate statement of the results?}
\begin{enumerate}[(a)]
\item At least one color of jelly bean is linked to acne.
\item At least one color of jelly bean is not linked to acne.
\item \solnMultOn{1}{2}{There is little evidence that any color of jelly bean is linked to acne.}
\item Jelly beans definitely do not cause acne.
\end{enumerate}

\pause \pause

\clicker{For the $F$-test, what is the probability of incorrectly rejecting the null?}
\begin{enumerate}[(a)]
\item \solnMultOn{3}{4}{5\%}
\item 36\%
\item 64\%
\item 95\%
\end{enumerate}

\end{frame}

%%%%%%%%%%%%%%%%%%%%%%%%%%%%%%

\begin{frame}

\vfill

\app{4.4 ANOVA - Pt 1}{See the course webpage for details.}

\vfill

\end{frame}

%%%%%%%%%%%%%%%%%%%%%%%%%%%%%%%%%%

\section{Summary}

%%%%%%%%%%%%%%%%%%%%%%%%%%%%%%%%%%%%

\begin{frame}
\frametitle{Summary of main ideas}

\vfill

\begin{enumerate}

\item \nameref{mi1}

\item \nameref{mi2}

\item \nameref{mi3}

\end{enumerate}

\vfill

\end{frame}

%%%%%%%%%%%%%%%%%%%%%%%%%%%%%%%%%%%

\end{document}