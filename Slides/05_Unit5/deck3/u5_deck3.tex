% -*- TeX-engine: xetex; eval: (auto-fill-mode 0); eval: (visual-line-mode 1); -*-
% Compile with XeLaTeX

%%%%%%%%%%%%%%%%%%%%%%%
% To do before class
%%%%%%%%%%%%%%%%%%%%%%%

% Send the Logistics/Week0Annoucnement (the night before).
% Send an email reminding students to bring a charged computer (the night before).

%%%%%%%%%%%%%%%%%%%%%%%
% Option 1: Slides: (comment for handouts)   %
%%%%%%%%%%%%%%%%%%%%%%%

% \documentclass[slidestop,compress,mathserif,12pt,t,professionalfonts,xcolor=table]{beamer}

% % solution stuff
% \newcommand{\solnMult}[1]{
% \only<1>{#1}
% \only<2->{\red{\textbf{#1}}}
% }
% \newcommand{\soln}[1]{\textit{#1}}

% \newcommand{\solnMultOn}[3]{
% \only<#1>{#3}
% \only<{#2}->{\red{\textbf{#3}}}
% }

%%%%%%%%%%%%%%%%%%%%%%%%%%%%%%%
% Option 2: Handouts, without solutions (post before class)    %
%%%%%%%%%%%%%%%%%%%%%%%%%%%%%%%

% \documentclass[11pt,containsverbatim,handout,xcolor=xelatex,dvipsnames,table]{beamer}

% % handout layout
% \usepackage{pgfpages}
% \pgfpagesuselayout{4 on 1}[letterpaper,landscape,border shrink=5mm]

% % solution stuff
% \newcommand{\solnMult}[1]{#1}
% \newcommand{\soln}[1]{}
% \newcommand{\solnMultOn}[3]{#3}

% % % This breaks things for me for some reason.
% % tell pgfpages how to set page sizes in XeLaTeX
% %\renewcommand\pgfsetupphysicalpagesizes{%
% %   \pdfpagewidth\pgfphysicalwidth\pdfpageheight\pgfphysicalheight%
% %}

%%%%%%%%%%%%%%%%%%%%%%%%%%%%%%%%%%%%
% Option 3: Handouts, with solutions (may post after class if need be)    %
%%%%%%%%%%%%%%%%%%%%%%%%%%%%%%%%%%%%

\documentclass[11pt,containsverbatim,handout,xcolor=xelatex,dvipsnames,table]{beamer}

% handout layout
\usepackage{pgfpages}
\pgfpagesuselayout{4 on 1}[letterpaper,landscape,border shrink=5mm]

% solution stuff
\newcommand{\solnMult}[1]{\red{\textbf{#1}}}
\newcommand{\soln}[1]{\textit{#1}}

% % This breaks things for me for some reason.
% % tell pgfpages how to set page sizes in XeLaTeX
% \renewcommand\pgfsetupphysicalpagesizes{%
%    \pdfpagewidth\pgfphysicalwidth\pdfpageheight\pgfphysicalheight%
% }

%%%%%%%%%%%%%%%%%%%%%%%%%%%%%%%
% Option 4: Notes Only
%%%%%%%%%%%%%%%%%%%%%%%%%%%%%%%

% % See http://tex.stackexchange.com/questions/11421/add-notes-to-latex-beamer
% \documentclass[10pt,containsverbatim,xcolor=xelatex,dvipsnames,table,notes=only]{beamer}

% % handout layout
% % \usepackage{pgfpages}
% % \pgfpagesuselayout{1 on 1}[letterpaper, landscape, border shrink=5mm]

% % solution stuff
% \newcommand{\solnMult}[1]{#1}
% \newcommand{\soln}[1]{}

% % % Having a problem with this.
% % tell pgfpages how to set page sizes in XeLaTeX
% % \renewcommand\pgfsetupphysicalpagesizes{%
% %   \pdfpagewidth\pgfphysicalwidth\pdfpageheight\pgfphysicalheight%
% %}

%%%%%%%%%%
% Load style file, defaults  %
%%%%%%%%%%

\input{../../lec_style.tex}
\input{../../definitions_default.tex}
% ALT ALT
% \input{../../definitions_custom.tex}

%%%%%%%%%%%
% Cover slide info    %
%%%%%%%%%%%

\title{Unit 5: Inference for categorical data}
\subtitle{3. Chi-square testing}
\author{Sta 101 - Spring 2015}
\date{March 23, 2015}
% ALT ALT
% \date{March 24, 2015}
\institute{Duke University, Department of Statistical Science}

%%%%%%%%%%%
% Begin document   %
%%%%%%%%%%%

\begin{document}

%%%%%%%%%%%%%%%%%%%%%%%%%%%%%%%%%%%%

% Title Page

\begin{frame}[plain]

\titlepage
\vfill
{\scriptsize \webLink{\PersonalSite}{Dr. \LastName{}} \hfill Slides posted at  \webLink{\CourseSite}{\CourseSite}}
\addtocounter{framenumber}{-1} 

\end{frame}

%%%%%%%%%%%%%%%%%%%%%%%%%%%%%%%%%%%%

\section{Housekeeping}

%%%%%%%%%%%%%%%%%%%%%%%%%%%%%%%%%%%%

\begin{frame}
\frametitle{Announcements}

\begin{itemize}

\item Mt2 Review tonight 7-8pm at Old Chem 116

\item This week: OH 3-5 today and Tuesday, support by TAs and on Piazza by me later in the week

\item Project due Saturday night

\item RA6 Monday (all videos, unit is shorter)

% ALT ALT

% \item Tomorrow in lab: work on Project 1---attendence is still mandatory.

% \item Project 1 due Monday at noon.

% \item RA6 Monday (all videos, unit is shorter)

\end{itemize}

\end{frame}

%%%%%%%%%%%%%%%%%%%%%%%%%%%%%%%%%%%%

\section{Main ideas}

%%%%%%%%%%%%%%%%%%%%%%%%%%%%%%%%%%%%

\subsection{Categorical data: 2 levels $\rightarrow$ Z, $>$2 levels $\rightarrow$ $\chi^2$ square}
\label{mi1}

%%%%%%%%%%%%%%%%%%%%%%%%%%%%%%%%%%%%

\begin{frame}
\frametitle{Inference for categorical data}

\textbf{If sample size related conditions are met:}

\pause

\begin{itemize}

\item Categorical data with 2 levels $\rightarrow$ Z

\pause

\begin{itemize}
\item one variable: Z HT  / CI for a single proportion
\item two variables: Z HT  / CI  comparing two proportions
\end{itemize} 

\pause

\item Categorical data with more than 2 levels $\rightarrow$ $\chi^2$

\pause

\begin{itemize}
\item one variable: \hl{$\chi^2$ test of goodness of fit}, no CI
\item two variables: \hl{$\chi^2$ test of independence}, no CI
\end{itemize} 

\end{itemize}

\pause
$\:$ \\

\textbf{If sample size related conditions are not met:} \pause Simulation based inference (randomization for HT / bootstrapping for CI, when appropriate)

\end{frame}

%%%%%%%%%%%%%%%%%%%%%%%%%%%%%%%%%%%

\begin{frame}

\clicker{In the basic Powerball game players select 5 numbers from a set of 59 white balls. We have historical data from lottery outcomes such that we are able to calculate how many times each of the 59 white balls were picked. We want to find out if each number is equally likely to be drawn. Which test is most appropriate?}

\begin{enumerate}[(a)]
\item Z test for a single proportion
\item Z test for comparing two proportions
\item \solnMult{$\chi^2$ test of goodness of fit}
\item $\chi^2$ test of independence
\end{enumerate}

\pause
$\:$ \\

\soln{\only<2|handout:0>{\red{ $H_0:$ Each outcome is equally likely, $p_1 = p_2 = \cdots = p_{59} = 1/59$ }}}

%---Note---%
\note{

People will be split between goodness of fit and $\chi^2$ test for dependence, so have them discuss and vote again.

}

\end{frame}

% ALT ALT - COMMENTED ABOVE ADDED BELOW 

% \begin{frame}

% % Another scenario to consider:
% % https://www.ucsf.edu/news/2014/10/119591/genetic-variant-protects-some-latina-women-breast-cancer

%   \clicker{You and a friend are playing craps, which relies on two dice.  Your friend brought the dice.  You have recorded the previously rolled totals as data.  Which test is most appropriate to check that the dice are fair?}

% \begin{enumerate}[(a)]
% \item Z test for a single proportion
% \item Z test for comparing two proportions
% \item \solnMult{$\chi^2$ test of goodness of fit}
% \item $\chi^2$ test of independence
% \end{enumerate}

% \pause
% $\:$ \\

% \soln{\only<2|handout:1>{\red{ $H_0:$ $p_2 = p_{12} = 1/36$; $p_{3} = p_{11} = 1/18$; $p_{4} = p_{10} = 1/12$; $p_{5} = p_{9} = 1/9$; $p_{6} = p_{8} = 5/36$; $p_{7} = 1/6$.}}}

% %---Note---%
% \note{

% People will be split between goodness of fit and $\chi^2$ test for dependence, so have them discuss and vote again.

% }

% \end{frame}

%%%%%%%%%%%%%%%%%%%%%%%%%%%%%%%%%%%

\begin{frame}

\clicker{A Gallup poll asked whether or not respondents identify as Tea Party Republican (yes / no) and whether or not they are motivated to vote in the upcoming midterm election (yes / no). We want to find out whether being a Tea Party Republican is associated with motivation to vote. Which test is most appropriate?}

\begin{enumerate}[(a)]
\item Z test for a single proportion
\item \solnMult{Z test for comparing two proportions}
\item $\chi^2$ test of goodness of fit
\item $\chi^2$ test of independence
\end{enumerate}

\pause
$\:$ \\

\soln{\only<2|handout:1>{\red{ $H_0: p_{TPR} = p_{Other}$, where $p =$ probability of being motivated to vote }}}

%---Note---%
\note{

It will be split between t-test for comparing two proportions and $\chi^2$ test of independence.  Mine has them vote again.

Write down table.

Explanation of why two prop vs. chisq: because two prop. is more conservative?

}

\end{frame}

%%%%%%%%%%%%%%%%%%%%%%%%%%%%%%%%%%%

\begin{frame}

\clicker{Suppose the Gallup poll instead asked about 
\begin{itemize}
\item party affiliation (Tea Party Republican, Other Republican, and Non-Republican), and 
\item motivation to vote (extremely unmotivated, very unmotivated, unmotivated, motivated, very motivated, extremely motivated) 
\end{itemize} 
We want to find out whether party affiliation is associated with motivation to vote. Which test is most appropriate?}

\begin{enumerate}[(a)]
\item Z test for a single proportion
\item Z test for comparing two proportions
\item $\chi^2$ test of goodness of fit
\item \solnMult{$\chi^2$ test of independence}
\end{enumerate}

\pause
$\:$ \\

\soln{\only<2|handout:1>{\red{ $H_0:$ Party affiliation and motivation to vote are independent }}}

\end{frame}

%%%%%%%%%%%%%%%%%%%%%%%%%%%%%%%%%%%

\subsection{The $\chi^2$ statistic is always positive and right skewed}
\label{mi2}

%%%%%%%%%%%%%%%%%%%%%%%%%%%%%%%%%%%

\begin{frame}
\frametitle{The $\chi^2$ statistic}

\hl{$\chi^2$ statistic:} When dealing with counts and investigating how far the observed counts are from the expected counts, we use a new test statistic called the \hl{chi-square ($\chi^2$) statistic}:
\[\chi^2 = \sum_{i = 1}^k \frac{(O - E)^2}{E} \qquad \text{where $k$ = total number of cells} \]

\textbf{Important points:}
\begin{itemize}
\item Use \textbf{counts} (not \textbf{proportions}) in the calculation of the test statistic, even though we're truly interested in the proportions for inference
\item Expected counts are calculated assuming the null hypothesis is true
\end{itemize}

\end{frame}

%%%%%%%%%%%%%%%%%%%%%%%%%%%%%%%%%%%

\begin{frame}
\frametitle{Expected Counts}

Example: does survival on the Titanic depend on cabin class?

\hfill \\

% See HypothesisTesting09.tex from Fall 2014
\begin{minipage}{0.5\textwidth}
Observed counts:
\begin{center}
\begin{tabular}{l c c c}
    & 1st & 2nd & 3rd \\
no  & 123 & 158 & 528 \\
yes & 200 & 119 & 181
\end{tabular}
\end{center}
\end{minipage}%
\begin{minipage}{0.5\textwidth}
Column props.:
\begin{center}
\begin{tabular}{l c c c}
    & 1st & 2nd & 3rd \\
  no  & 0.38 &  0.57 & 0.74 \\
  yes & 0.62 & 0.43 & 0.26
\end{tabular}
\end{center}
\end{minipage}

\pause

\hfill \\

Intuition:
\[
E_{\text{no, 1st}} = \hat p_{no} \times \hat p_{1st} \times \text{(total \# obs)}
\]

Simplification:
\[
E_{\text{no, 1st}} = \frac{\text{row ``no'' total} \times \text{col ``1st'' total}}{\text{table total}} = \frac{809 \times 323}{1309} = 199.6
\]

\pause

$\chi^2_{titanic} = 127.9$

\end{frame}


%%%%%%%%%%%%%%%%%%%%%%%%%%%%%%%%%%%

\begin{frame}
\frametitle{The $\chi^2$ distribution}

The $\chi^2$ distribution has just one parameter, \hl{degrees of freedom (df)}, which influences the shape, center, and spread of the distribution.
\begin{itemize}
\item For $\chi^2$ GOF test: $df = k - 1$ \\
\item For $\chi^2$ independence test: $df = (R-1) \times (C-1)$ 
\end{itemize}

\pause

\begin{center}
\includegraphics[width=0.8\textwidth]{figures/chiSquareDistributionWithInceasingDF/chiSquareDistributionWithInceasingDF}
\end{center}

\end{frame}

%%%%%%%%%%%%%%%%%%%%%%%%%%%%%%%%%%

\begin{frame}[fragile]
\frametitle{Finding areas under the chi-square curve}

p-value = tail area under the chi-square distribution (as usual)

\pause

\begin{itemize}

\item Using the applet: \webURL{http://bit.ly/dist_calc}

\pause

\item Using R: \texttt{pchisq()}

\pause

\item Using the table: works a lot like the $t$ table, but only provides upper tail values.
\end{itemize}

\begin{center}
\includegraphics[width=0.3\textwidth]{figures/above5Point1WithDF5/above5Point1WithDF5}
\end{center}

{\scriptsize
\begin{center}
\begin{tabular}{r | rrrr | rrrr |}
  \hline
Upper tail & 0.3 & 0.2 & 0.1 & 0.05 & 0.02 & 0.01 & 0.005 & 0.001 \\ 
  \hline
df \hfill 1 &  1.07 &  1.64 &  2.71 &  3.84 &  5.41 &  6.63 &  7.88 &  10.83 \\ 
  2 &  2.41 &  3.22 &  4.61 &  5.99 &  7.82 &  9.21 &  10.60 &  13.82 \\ 
  3 &  3.66 &  4.64 &  6.25 &  7.81 &  9.84 &  11.34 &  12.84 &  16.27 \\ 
  4 &  4.88 &  5.99 &  7.78 &  9.49 &  11.67 &  13.28 &  14.86 &  18.47 \\ 
  5 &  6.06 &  7.29 &  9.24 &  11.07 &  13.39 &  15.09 &  16.75 &  20.52 \\ 
  \hline
  6 &  7.23 &  8.56 &  10.64 &  12.59 &  15.03 &  16.81 &  18.55 &  22.46 \\ 
  $\cdots$ &   &   &   &   &   &   &   &   \\ 
\end{tabular}
\end{center}
}

\end{frame}

%%%%%%%%%%%%%%%%%%%%%%%%%%%%%%%%%%%

\begin{frame}
\frametitle{Computing a $p$-value using the table}

\clicker{In the Titanic example, $\chi^2_{titanic} = 127.9$ and $df=2$.  Based on the table from the previous slide, which of the following is correct?  (Hint: draw a picture!)}

  \begin{enumerate}[(a)]
  %\item The $p$-value for this data set will be in the interval $(0.2, 0.3]$.
  %\item The $p$-value for this data set will be in the interval $(0.1, 0.2]$.
   %\item The $p$-value for this data set will be in the interval $(0.05, 0.1]$.
  \item The $p$-value for this data set will be in the interval $(0.02, 0.05]$.
  \item The $p$-value for this data set will be in the interval $(0.01, 0.02]$.
  \item The $p$-value for this data set will be in the interval $(0.005, 0.01]$.
  \item The $p$-value for this data set will be in the interval $(0.001, 0.005]$.
  \item \solnMult{The $p$-value for this data set will be at or below $0.001$.}
  \end{enumerate}

\end{frame}

%%%%%%%%%%%%%%%%%%%%%%%%%%%%%%%%%%%

\begin{frame}
\frametitle{Interpretation}

\clicker{What is the best interpretation of the hypothesis test?}

  \begin{enumerate}[(a)]
    \item There is not convincing evidence that survival and cabin class are dependent.
    \item \solnMult{There is convincing evidence that survival and cabin class are dependent.}
  \end{enumerate}

\end{frame}


%%%%%%%%%%%%%%%%%%%%%%%%%%%%%%%%%%%

\subsection{At least 5 expected successes for $\chi^2$ testing}
\label{mi3}

%%%%%%%%%%%%%%%%%%%%%%%%%%%%%%%%%%%

\begin{frame}
\frametitle{Conditions for $\chi^2$ testing}

\begin{enumerate}

\item \hlGr{Independence:} In addition to what we previously discussed for independence, each case that contributes a count to the table must be independent of all the other cases in the table.

\item \hlGr{Sample size / distribution:} Each cell must have at least 5 \red{expected} cases.

\end{enumerate}

\end{frame}

%%%%%%%%%%%%%%%%%%%%%%%%%%%%%%%%%%%

\begin{frame}

\clicker{Suppose a poll asked the following questions:
\begin{itemize}
\item How would you identify your socio-economic status: low, middle, high?
\item What type of pet did you have growing up, select all that apply: cat, dog, fish, bird, rodent, none of the above?
\end{itemize}
What test is most appropriate for evaluating the relationship between these two variables?}

\begin{enumerate}[(a)]
\item Z test for a single proportion
\item Z test for comparing two proportions
\item $\chi^2$ test of goodness of fit
\item $\chi^2$ test of independence
\item \solnMult{none of the above}
\end{enumerate}

%---Note---%
\note{

They will mess this up once.  Have them vote again.

You must have disjoint categories.

}

\end{frame}

% ALT ALT - COMMENTED ABOVE

%%%%%%%%%%%%%%%%%%%%%%%%%%%%%%%%%%%

\section{Application exercises}

%%%%%%%%%%%%%%%%%%%%%%%%%%%%%%%%%%%

\begin{frame}

\vfill

\app{5.3 Chi-square tests}{See course website for details.}

\vfill

%---Note---%
\note{

25 minutes for app ex.

If you aren't done in first 10 minutes with part 1, raise your hand.

}

\end{frame}


%%%%%%%%%%%%%%%%%%%%%%%%%%%%%%%%%%%

% ALT ALT - ADDED SECTION ON RANDOMIZATION TEST

% \section{Randomization Test}

% \begin{frame}
% \frametitle{Recap: Does smoking habit depend on exercise habit?}

% Does smoking habit depend on exercise habit?
% \begin{center}
% \begin{tabular}{c c c c}
%           &  Freq.Exer & Not.Freq.Exer & Total \\
%   Non.Smoker &      87  &         102 & 189 \\
%   Smoker     &      28    &       19  & 47  \\
%   Total      &      115  & 121 & 236
% \end{tabular}
% \end{center}

% \begin{itemize}
% \item $H_0$: $p_{\text{freq. exer}} = p_{\text{not freq. exer}}$
% \item $H_A$: $p_{\text{freq. exer}} \neq p_{\text{not freq. exer}}$
% \end{itemize}

% \end{frame}

% %%%%%%%%%%%%%%%%%%%%%%%%%%%%%%%%%%%
% \begin{frame}
% \frametitle{Randomization test for the difference of two proportions}

% \begin{enumerate}

% \item Use 236 index cards, where each card represents an observation.

% \pause

% \item Mark 189 of the cards as ``Non Smoker'' and the remaining 47 as ``Smoker.''

% \pause

% \item Shuffle the cards and split into two groups of size of size 115 and 121 corresponding to ``Freq.\ Exer'' and ``Not Freq. Exer'' respectively.

% \pause

% \item Calculate the difference between the proportions of ``Non Smoking" in the frequent exercise and not frequent exercise groups, and record this number.

% \pause

% \item Repeat steps (3) and (4) many times to build a randomization distribution of differences in simulated proportions.

% \end{enumerate}

% \end{frame}


% %%%%%%%%%%%%%%%%%%%%%%%%%%%%%%%%%%%

% \begin{frame}
% \frametitle{Recap: Does smoking habit depend on exercise habit?}

% Does smoking habit depend on exercise habit?
% \begin{center}
% \begin{tabular}{c c c c c}
%        & Freq &None & Some & Total \\
%   Heavy&   7  & 1   &3  & 11 \\
%   Never&  87  & 18  &84 & 189 \\
%   Occas&  12  & 3   &4  & 19 \\
%   Regul&   9  & 1   &7  & 17 \\
%   Total& 115  & 23  &98 &236
% \end{tabular}
% \end{center}

% \begin{itemize}
% \item $H_0$: smoking habits are not dependent on exercise habits.
% \item $H_A$: smoking habits are depenednet on exercise habits.
% \end{itemize}

% \end{frame}

% %%%%%%%%%%%%%%%%%%%%%%%%%%%%%%%%%%%
% \begin{frame}
% \frametitle{Randomization test for the dependence of two categorical variables}

% \begin{enumerate}

% \item Use 236 index cards, where each card represents an observation.

% \pause

% \item Mark 11 of the cards ``Heavy'', 189 of the cards ``Never'', 19 of the cards ``Occas'', 17 of the cards ``Regul''.

% \pause

% \item Shuffle the cards and split into 3 groups of size of size 115, 23, and 98 corresponding to ``Freq'', ``None'', and ``Some'' respectively.

% \pause

% \item Calculate the the $\chi^2$ test statistic for the shuffled data.

% \pause

% \item Repeat steps (3) and (4) many times to build a randomization distribution of many $\chi^2$ test statistics.

% \end{enumerate}

% \end{frame}

% %%%%%%%%%%%%%%%%%%%%%%%%%%%%%%%%%%%
% \begin{frame}
% \frametitle{Randomization test for the dependence of two categorical variables}

% \vfill

% Check out \webLink{https://stat.duke.edu/~jbw44/Teaching/Spring2015/post/misc/chisq-randomization.R.txt}{chisq-randomization.R} on the course website.

% \vfill

% \end{frame}



%%%%%%%%%%%%%%%%%%%%%%%%%%%%%%%%%%%

\section{Summary}

%%%%%%%%%%%%%%%%%%%%%%%%%%%%%%%%%%%

\begin{frame}
\frametitle{Summary of main ideas}

\vfill

\begin{enumerate}

\item \nameref{mi1}

\item \nameref{mi2}

\item \nameref{mi3}

\end{enumerate}

\vfill

\end{frame}

%%%%%%%%%%%%%%%%%%%%%%%%%%%%%%%%%%%

\end{document}